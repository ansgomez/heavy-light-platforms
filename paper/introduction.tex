%In the high performance embedded systems domain, advanced energy minimization techniques (e.g. fine-grained DVFS) can be used to achieve lower power consumption. 
%Unfortunately, these techniques cannot be employed in low-cost, power-constrained microcontroller units (MCUs), due to the area and energy overhead required to enable these techniques. New methods are needed to satisfy the demand for even more efficient and better performing MCUs. 

%However, power-constrained MCUs can improve both their responsiveness and their energy consumption by adding co-processors with reduced design margins, at the cost of increased sensitivity to process, voltage and temperature (PVT) variations. 

%In this work, we derive energy-optimal and performance-oriented task allocation policies for such a platform and we study their impact by emulating the runtime environment using the \emph{LPCXpresso54102} platform. 

%\section{Motivation}
%Symmetric multi-core architectures are traditionally used as a means of achieving increased performance and reduced power consumption, compared to single-core architectures. The use of multiple cores allows each core to run at a lower frequency and voltage, thus significantly reducing the power. Other techniques for reducing power have also been investigated. \emph{Asychronous Clock Architectures} is another promising technique for multi-core processors \cite{ARM3}, in which each core is operated at different voltage level and clock frequency. A sophisticated scheduler is mandatory in this architecture, so that each task can be allocated to the appropriate processor, depending on its computational load. Asychronous Clock Architectures try to achieve a balance between performance and energy consumption.\par

Over the past decade, there have been a large research effort to reduce the energy consumption in embedded systems. Heterogeneous architectures have emerged as a viable candidate in high performance embedded systems. One example is the heterogeneous big.LITTLE architecture \cite{ARM}, which consists of a big cluster, for performance, and a LITTLE cluster, for energy efficiency. Even though the two clusters are microarchitecturally different, they support the same instruction set architecture (ISA), simplifying their programming. In order to fully exploit their energy saving potential, these architectures require an advanced HW/SW infrastructure, such as fine-grained Dynamic Voltage and Frequency Scaling (DVFS), memory virtualization, multi-threading, thread migration and a full-featured operating system. All of these are beyond the reach of low-end MCUs. %\par

%Their difference lies on the fact that the LITTLE cluster consists of cores which are simple, in-order, with a pipeline of between 8-10 stages, whereas the big cluster has cores which are superscalar, out-of-order, with a pipeline of between 15-25 stages. The main idea behind the big.LITTLE architecture is to dynamically allocate the tasks to the right cluster, depending on how demanding they are. The highly demanding tasks should be allocated to the big cluster, whereas the less demanding ones to the LITTLE cluster. The scheduler constantly monitors the task load and determines to which cluster each task should be allocated to. Therefore, it is a combination of symmetric multiprocessing and asychronous clock architectures and uses advanced techniques, such as Dynamic Voltage and Frequency Scaling (DVFS), memory virtualization, multi-threading and a full-featured operating system, to further improve the energy efficiency.\par

On the other end of the spectrum, the low-end microcontroller market is currently dominated by simple, cache-less, single-core platforms optimized for low power and cost. Those advanced energy saving techniques from high-end systems are not supported. Consequently, these systems typically rely on frequency scaling and shutdown for minimizing their energy consumption. More recently, two important trends in the microcontroller domain have opened new avenues for tackling the energy consumption problem. First, heterogeneous dual-core architectures have been shown to reduce energy when compared to single-core platforms~\cite{Fuks}. These systems, with different microarchitectures and ISAs, can exhibit high energy efficiency with certain applications. However, one limitation is that due to different ISAs, tasks are (statically) allocated at design time, limiting dynamic (online) decisions.% Therefore, dynamic task allocation decisions cannot be made.  %If, for example, an application consists of both computationally demanding and control/data handling tasks, the former can be statically mapped to a more powerful core like a Cortex M4F, while the latter to a low-power M0+, introducing significant energy savings. If a task set consists purely of computationally demanding tasks, as is commonly the case in deeply embedded systems, they can introduce significant energy, cost, or area overheads. Consequently, both the application and the task allocation/mapping have a significant impact on the system's behavior, both in terms of energy and response times. \par
%One the biggest limitations of these platforms is that the tasks have to be (statically) allocated at design time, since processors have different ISA's, and it is not possible to implement complex feedback allocation policies, due to their limited performance.

To circumvent this limitation, the semiconductor industry has started focusing on dual-core platforms with different model guardbands for different cores. Reduced guardbands lead to reduced core area and increased energy efficiency. %Researchers have been studying the impact of process variation in such cores and have proposed techniques to deal with its problems \cite{Kahng, Paterna}.
Recent studies have shown that a dual-core platform consisting of one (Heavy) core designed to work reliably under the worst-case conditions and another (Light) core designed using reduced design margins, can reduce the platform's energy consumption by up to 20\% compared to a single-core designed for worst-case conditions\cite{Gomez1}. Due to uniform ISA, task allocation in these platforms does not have to be static/fixed. 

In this work, we study the tradeoff between minimizing energy and maximizing performance when executing an application on a Heavy-Light platform. Heavy-Light platforms are restrictive in terms of the energy saving mechanisms available:no DVFS or fine-grained power gating can be applied. %Therefore, DVFS is not used in our solution. 
As is typical in microcontroller-based platforms, there is also no multi-threading or thread migration due to their excessive cost. Therefore, our proposed solutions are task allocation policies geared towards minimizing energy or maximizing performance. We prove that for Heavy-Light platforms, there is an assignment of tasks to cores, which minimizes both energy consumption and application makespan. Therefore, for the optimal case, there is no tradeoff between energy and performance. Furthermore, we prove that even when the optimal task assignment/partitioning cannot be achieved, there is no tradeoff between energy and performance when task level parallelism is high. We evaluate the proposed concepts by detailed simulations and evaluation on a hardware testbed.  

%The rest of this paper is organized as follows: Section~\ref{related} covers the related research in energy minimization strategies followed by the system and task model in Section~\ref{system_task_model}. Section~\ref{theory} presents the theoretical results and proposed task allocation policies that optimize energy consumption and/or performance. Results are presented in Section~\ref{results}. This section also includes the evaluation of the proposed schemes on a hardware test-bed. This is followed by conclusion and references. 


%We identify that for many system configurations and workloads, the two objectives are non-conflicting. Therefore, there is no tradeoff

%As opposed to single-ISA heterogeneous architectures, Heavy-Light platforms need only simple task allocation schemes and do not require advanced hardware/software infrastructure to introduce energy savings. Compared to multi-ISA heterogeneous architectures, Heavy-Light platforms are well-suited for homogeneous workloads (where one ISA is energy optimal). Due to their novelty, there are no works on energy and makespan minimization on Heavy-Light platforms. The contributions of this work can be summarized as follows: we

%\begin{itemize}
%	\item Propose a simplified task model and conditions where energy and makespan can be simultaneously minimized
%	\item Develop and formally prove the optimality of a task allocation policy with respect to energy and makespan
%	\item Validate theoretical results using a modified commercially available platform.
%\end{itemize}

%We study these heavy-light platforms in this project and show that the proposed architecture can result in energy savings up to 24\%, depending on the workload and the platform characteristics.

% Therefore, the two cores can execute the same applications, which allows the development of more sophisticated task allocation policies and overcomes the problem of the static allocation of tasks at design time. We study these heavy-light platforms in this project and show that the proposed architecture can result in energy savings up to 24\%, depending on the workload and the platform characteristics.