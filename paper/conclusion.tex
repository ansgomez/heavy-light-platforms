%%%%Conclusion%%%

In this work, we present a task allocation scheme which targets makespan and energy minimization, for the recently proposed Heavy-Light platform. We prove that only if the level is task-level parallelism is high enough, are the two objectives of minimizing makespan and minimizing energy equal. %Therefore, we do not have to tradeoff performance for energy efficiency. 
For scenarios where task level parallelism is low, we propose two Delta Threshold Mapping (DTM) policies which target performance (DTM-M) and energy (DTM-E). We theoretically prove the optimality of the proposed allocation policies and evaluate them on a hardware testbed. The results show an improvement of up to 50\% in makespan and up to 22\% in energy consumption; compared to a single-core platform.

%The current trends towards heterogeneous multi-core platforms and guardbands reductions, have opened new ways to address the energy consumption problem of MCUs. The recently proposed Heavy-Light architecture converges these trends with a platform capable of reducing the energy and makespan by only exploiting task-level parallelism. Furthermore, we have proven the optimality of our proposed task allocation scheme with respect to energy and makespan. For applications with enough available parallelism, our proposed policy can simultaneously minimize the application's energy and makespan. This optimality bound was proven mathematically and verified experimentally. Compared to state of the art policies on Heavy-Light architectures, our proposed policy holds valid for a wide range of operating frequencies, reaching up to 50\% in makespan reduction and up to 22\% in energy savings.

%Recent works have proposed hybrid architectures, consisting of one core designed to work reliably, independently on the environmental conditions (Heavy Core), and another designed to work under more relaxed conditions (Light Core). This work studies the impact of such an architecture on the performance and the energy consumption. We investigated different allocation scenarios and derived policies focusing both on reducing the makespan and maximizing the energy savings. Furthermore, we supported our proposals with mathematical proofs and validated the theoretical results by using emulation. Our results show substantial improvements, both in performance and energy consumption, reaching up to 50\% in makespan reduction and up to 24\% in energy savings, depending on the workload and platform characteristics.
%This work studies the impact of recently proposed hybrid architectures on the performance and energy consumption. We investigated different allocation scenarios and derived policies focusing both on reducing the makespan and maximizing the energy savings. Furthermore, we supported our proposals with mathematical proofs and validated the theoretical results by using emulation.